\documentclass{article}

\title{Coins-Ex}
\author{Filippo Di Pietro}
\date{August 2022}


\begin{document}
\pagenumbering{gobble}
\begin{titlepage}
    \begin{center}
        \vspace*{1cm}
            
        \Huge
        \textbf{Coins-Ex}
            
        \vspace{0.5cm}
        \LARGE
        Più exchange di criptovalute
            
        \vspace{1.5cm}
            
        \textbf{Filippo Di Pietro}
            
        \vfill
        August 2022
            
    \end{center}
\end{titlepage}
\newpage
\pagenumbering{arabic}
\section{Analisi dei requisiti}
Si vuole realizzare un database di più Exchange di cryptovalute.
Lo scopo di un Exchange è dare la possibilità agli utenti di poter scambiare valute legali ( Euro, Dollaro, etc...) in cryptovalute, tenendo traccia degli scambi tra una valuta ed un’altra, e mostrarne un grafico.
\subsection{Intervista}
Si vuole tenere traccia dei dipendenti e supervisori, che lavorano in un determinato exchange, questi lavoratori posso anche essere registrati come utenti nell’ Exchange in cui lavorano.
Si vuole poi tenere traccia degli utenti che si registrano agli exchange, memorizzando: Nome, Cognome, Email, Data di nascita, Codice fiscale, Nazionalità, numero di telefono, Password (in hash). Ad ogni utente al momento della registrazione viene creato un primo conto corrente, contenente la valuta FIAT (valuta legale) del paese dell’ utente registrato, dove potrà depositare i suoi primi soldi, poi eventualmente potrà creare più conti correnti sempre della stessa valuta oppure anche di valute diverse ma sempre FIAT. Ogni entità conto può effettuare transazioni che posso essere o in entrata o in uscita, di conseguenza pure Conto Corrente e wallet ( Conto che contiene solo crypto ) che sono sotto entità. Appena un utente vuole fare uno scambio, es. 1000\$ per 0.1 Bitcoin, deve effettuare un ordine, ed aspettare che qualcuno faccia l’ordine speculare, ossia: possiede 0.1 Bitcoin e voglia 1000\$. Appena viene effettuato lo scambio vengono create 2 transazioni ( una che sposta 1000\$ da un conto corrente ad un altro, e uno che sposta 0.1 BTC in un altro wallet). Tutte le transazioni (prese 2 a 2) però non corrispondono per forza ad uno scambio e quindi a due ordini, poiché un utente può inviare soldi ad un’ altro utente senza avere una valuta in cambio. La super entità Valuta contiene due attributi il Ticker che è del testo, univoco a cui corrisponde la valuta (es. EUR, USD; BTC, ...) e il nome completo di tale valuta. Un Exchange inoltre possiede degli ATM, i quali permettono di depositare o prelevare qualsiasi moneta FIAT dando in cambio qualsiasi cryptovaluta, però tale scambio è differente da quello che avviene sugli exchange poiché quelli che avvengono negli ATM vi è una commissione “nascosta” chiamata spread, che può variare da differenti ATM e anche per lo stesso ATM può variare nel tempo, ad ogni prelievo o deposito viene aggiunta una entità transazione fisica che tiene traccia della valuta FIAT prelevata o depositata, in più sarà collegata una transazione di cryptovalute, poiché sia in caso di prelievo che deposito, di denaro vi è pure una transazione di cryptovaluta tra l’utente e l’ATM. Scegliendo un exchange specifico il software quindi deve calcolare, il prezzo medio di scambio di una valuta con un’altra, farne un grafico, calcolare il numero medio di scambi, oltre che interfacciarsi con il DBMS per ottenere informazioni di più exchange.
\newpage

\subsection{Rilevamento delle ambiguità e correzioni proposte}

\newpage
\subsection{Definizione delle specifiche in linguaggio naturale ed estrazione dei concetti principali}

\newpage
\section{Progettazione Concettuale}

\newpage
\section{Progettazione logica}

\newpage
\section{Progettazione dell'applicazione}
\end{document}
